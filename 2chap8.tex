\chapter{Canonical Quantization of the Dirac Field}
\section{Conjugate momenta and canonical anti-comutation relations}
Now we canonically quantize the Dirac Field. We calculate the conjugate momentum
\begin{equation}
    \Pi^\beta=\pdv{\mathcal{L}}{(\partial_0\Psi_\beta)}
\end{equation}
from the Dirac Field Lagrangian (\ref{dirac_lagrangian}) $$\mathcal{L}=i\Bar{\Psi}\gamma^\mu\partial_\mu\Psi-m\Bar{\Psi}\Psi$$
and get
\begin{equation}
    \Pi^\beta=i\Bar{\Psi}_\delta(\gamma^0)^{\delta\beta}.
    \label{can_mom_dirac}
\end{equation}
The canonical anti-commutation relations 
\begin{equation*}
    \begin{aligned}
        \pb{\Psi_\alpha(t,\vb{x})}{\Psi_\beta(t,\vb{y})}&=0\\
        \pb{\Psi_\alpha(t,\vb{x})}{\Pi^\beta(t,\vb{y})}&=i\delta_\alpha^\beta\delta^3(\vb{x}-\vb{y})
    \end{aligned}
\end{equation*}
with canonical momentum (\ref{can_mom_dirac}) and bearing in mind that $(\gamma^0)^2=I$ read
\begin{equation}
    \begin{aligned}
        \pb{\Psi_\alpha(t,\vb{x})}{\Psi_\beta(t,\vb{y})}&=0\\
        \pb{\Psi_\alpha(t,\vb{x})}{\Bar{\Psi}_\beta(t,\vb{y})}&=(\gamma^0)_{\alpha\beta}\delta^3(\vb{x}-\vb{y})
    \end{aligned}
    \label{anticomm_dirac}
\end{equation}
\section{The Dirac field mode expansion}
Now we focus on solving the equation of motion, the Dirac Equation:
\begin{equation*}
    (-i\slashed{\partial}+m)\Psi=0.
\end{equation*}
Where we are introducing the \textit{Feynman slash} notation to indicate scalar products between 4-vectors and gamma matrices, that is, $\slashed{\partial}=\gamma^\mu\partial_\mu$. To get insight about the solutions to the Dirac equation, note that acting with $(i\slashed{\partial}+m)$ on it gives
\begin{equation}
    \begin{aligned}
        (\slashed{\partial}\slashed{\partial}+m^2)\Psi=0.
    \end{aligned}
    \label{dirac_reit}
\end{equation}
Now, since $\slashed{\partial}\slashed{\partial}=\gamma^\mu\gamma^\nu\partial_\mu\partial_\nu$ and $\partial_\mu$ and $\partial_\nu$ commute, we can replace $\gamma^\mu\gamma^\nu$ by its symmetric part $\frac{1}{2}\pb{\gamma^\mu}{\gamma^\nu}=-g^{\mu\nu}$. So $\slashed{\partial}\slashed{\partial}=- \partial^2$  and (\ref{dirac_reit}) is nothing but the Klein-Gordon equation. This tells us that $\Psi$ is also a solution to the Klein-Gordon equation, so it can be expanded in terms of plane-wave modes
\begin{equation}
    \Psi(x)=u(\vb{p})e^{ipx}+v(\vb{p})e^{-ipx}
\end{equation}
where $u(\vb{p})$ and $v(\vb{p})$ are four-component constant bispinors. Plugging this ansatz into the Dirac Equaton gives $(\slashed{p}+m)u(\vb{p})e^{ipx}+(-\slashed{p}+m)v(\vb{p})e^{-ipx}=0$ so we must simultaneously satisfy
\begin{equation}
\begin{aligned}
     (\slashed{p}+m)u(\vb{p})&=0\\
     (-\slashed{p}+m)v(\vb{p})&=0.
\end{aligned}
\label{uv}
\end{equation}
These two equations have two linearly independent solutions $u_{\pm}(\vb{p})$ and $v_{\pm}(\vb{p})$ so the mode expansion for the Dirac field must be
\begin{equation}
    \Psi(x)=\sum_{s=\pm}\int\widetilde{\dd p}\qty[b_s(\vb{p})u_s(\vb{p})e^{ipx}+d^\dagger_s(\vb{p})v_s(\vb{p})e^{-ipx}]
    \label{dirac_mode_expansion}
\end{equation}
where, as usual, $\widetilde{\dd p}=\frac{\dd^3 p}{(2\pi)^3 2\omega}$ is the Lorentz Invariant integration measure, $\omega=\sqrt{\vb{p}^2+m^2}$, $b_s(\vb{p})$ and $d^\dagger_s(\vb{p})$ are coefficients to be interpreted as creation and annihilation operators upon quantization.\\

As for the solutions to (\ref{uv}), we label them in the rest frame $\vb{p}=\vb{0}$ by $u_{\pm}(\vb{0})$ and $v_{\pm}(\vb{0})$, and we distinguish  them by their spin along the $z$-axis, indicated by their eigenvalues upon the action of $S_z$
\begin{equation}
S_{z}=\frac{i}{4}\left[\gamma^{1}, \gamma^{2}\right]=\frac{i}{2} \gamma^{1} \gamma^{2}=\left(\begin{array}{cc}
\frac{1}{2} \sigma_{3} & 0 \\
0 & \frac{1}{2} \sigma_{3}
\end{array}\right).
\end{equation}
We require 
\begin{equation}
    \begin{aligned}
         S_z u_{\pm}(\vb{0})&=\pm\frac{1}{2}u_{\pm}(\vb{0})\\
         S_z v_{\pm}(\vb{0})&=\mp\frac{1}{2}v_{\pm}(\vb{0}).
    \end{aligned}
\end{equation}
A choice that guarantees that both $b^\dagger_{+}$ and $d^\dagger_{+}$ create spin up particles along the $z$-axis \cite{srednicki2007quantum}. In the rest frame $\slashed{p}=p_\mu\gamma^\mu=p_0\gamma^0=-m\gamma^0$ so equations (\ref{uv}) reduce to
\begin{equation}
\begin{aligned}
    m(-\gamma^0+I_4)u_s(\vb{0})&=m\left(\begin{array}{cc}
I_2 & -I_2 \\
-I_2 & I_2
\end{array}\right)u_s(\vb{0})=0\\
    m(\gamma^0+I_4)v_s(\vb{0})&=m\left(\begin{array}{cc}
I_2 & I_2 \\
I_2 & I_2
\end{array}\right)u_s(\vb{0})=0
\end{aligned}
\end{equation}
where $I_n$ is the $n\times n$ identity matrix. The solutions are
\begin{equation}
\begin{aligned}
    u_{+}(\mathbf{0})=\sqrt{m}\left(\begin{array}{l}
1 \\
0 \\
1 \\
0
\end{array}\right)\qquad & u_{-}(\mathbf{0})=\sqrt{m}\left(\begin{array}{c}
0 \\
1 \\
0 \\
1
\end{array}\right) \\
v_{+}(\mathbf{0})=\sqrt{m}\left(\begin{array}{c}
0 \\
1 \\
0 \\
-1
\end{array}\right) \qquad & v_{-}(\mathbf{0})=\sqrt{m}\left(\begin{array}{c}
-1 \\
0 \\
1 \\
0
\end{array}\right) .
\end{aligned}
\label{uv_columm}
\end{equation}
As for the Barred spinors
\begin{equation}
\begin{aligned}
    \Bar{u}_s(\vb{p})=&u_s^\dagger(\vb{p})\beta\\
    \Bar{v}_s(\vb{p})=&v_s^\dagger(\vb{p})\beta,
\end{aligned}
\end{equation}
we have
\begin{equation}
    \begin{aligned}
        \Bar{u}_{+}(\vb{0})&=\sqrt{m}(1,0,1,0)\\
        \Bar{u}_{-}(\vb{0})&=\sqrt{m}(0,1,0,1)\\
        \Bar{v}_{+}(\vb{0})&=\sqrt{m}(0, -1, 0,1)\\
        \Bar{v}_{-}(\vb{0})&=\sqrt{m}(1 , 0 , -1 , 0).
    \end{aligned}
    \label{uv_row}
\end{equation}
Which are the solutions to the barred (\ref{uv}) equations
\begin{equation}
    \begin{aligned}
        \Bar{u}_s(\vb{p})(\slashed{p}+m)&=0\\
        \Bar{v}_s(\vb{p})(-\slashed{p}+m)&=0
    \end{aligned}
    \label{uv_bar}
\end{equation}
since $\overline{(\slashed{p}+m)u_s(\vb{p})}=[u_s^\dagger(\vb{p})(\slashed{p}+m)]\beta=u_s^\dagger(\vb{p})\beta\beta^{-1}(\slashed{p}+m)\beta=\bar{u}_s(\vb{p})\beta(\slashed{p}+m)\beta=\Bar{u}_s(\vb{p})(\slashed{p}+m)$ where we recall that $\beta^{\text{T}}=\beta^\dagger=\beta^{-1}=\beta$ and highlight that for a general combination of gamma matrices $A$, $\Bar{A}=\beta A^\dagger\beta$.
\section{Important spinor relations}
Knowing $u_s(\vb{0})$ and $v_s(\vb{0})$, we can obtain $u_s(\vb{p})$ and $v_s(\vb{p})$ using a boost in the direction $\vu{p}=\vb{p}/\abs{\vb{p}}$, with rapidity $\eta=\abs{\vb{p}}/m$. Given the boost generator $K^j=\frac{i}{2}\gamma^j\gamma^0$, we consider the transformation
\begin{equation}
    D(\Lambda)=\exp(i\eta\vu{p}\cdot\vb{K})
\end{equation}
to act on $u_s(\vb{0})$ and $v_s(\vb{0}):$
\begin{equation}
    \begin{aligned}
        u_s(\vb{p})&=\exp(i\eta\vu{p}\cdot \vb{K})u_s(\vb{0})\\
        v_s(\vb{p})&=\exp(i\eta\vu{p}\cdot \vb{K})v_s(\vb{0}).
    \end{aligned}
\end{equation}
Barring these gives
\begin{equation}
    \begin{aligned}
        \bar{u}_s(\vb{p})&=\bar{u}_s(0)\exp(-i\eta\vu{p}\cdot \vb{K})\\
        \bar{v}_s(\vb{p})&=\bar{v}_s(0)\exp(-i\eta\vu{p}\cdot \vb{K}).
    \end{aligned}
\end{equation}
So, from these results and from (\ref{uv_columm}) and (\ref{uv_row}), we can check that
\begin{equation}
\begin{array}{l}
\bar{u}_{s^{\prime}}(\mathbf{p}) u_{s}(\mathbf{p})=+2 m \delta_{s^{\prime} s} \\
\bar{v}_{s^{\prime}}(\mathbf{p}) v_{s}(\mathbf{p})=-2 m \delta_{s^{\prime} s} \\
\bar{u}_{s^{\prime}}(\mathbf{p}) v_{s}(\mathbf{p})=0 \\
\bar{v}_{s^{\prime}}(\mathbf{p}) u_{s}(\mathbf{p})=0.
\end{array}
\end{equation}

Also, noting that
\begin{equation}
    \begin{aligned}
        \gamma^\mu\slashed{p}&=\frac{1}{2}\pb{\gamma^\mu}{\slashed{p}}+\frac{1}{2}\comm{\gamma^\mu}{\slashed{p}}=-p^\mu-2iS^{\mu\nu}p_\nu\\
        \slashed{p}^\prime\gamma^\mu&=\frac{1}{2}\pb{\gamma^\mu}{\slashed{p}^\prime}-\frac{1}{2}\comm{\gamma^\mu}{\slashed{p}^\prime}=-p^{\prime\mu}+2iS^{\mu\nu}p^\prime_\nu,
    \end{aligned}
\end{equation}
adding the first to the second, sandwiching them between $\bar{u}_{s^\prime}(\vb{p}^\prime)$ and $u_s(\vb{p})$, noting from (\ref{uv}) that $\slashed{p}u_s(\vb{p})=-mu_s(\vb{p})$ and from (\ref{uv_bar}) that $\bar{u}_{s^\prime}(\vb{p}^\prime)\slashed{p}=-\bar{u}_{s^\prime}(\vb{p}^\prime)m$, we find
\begin{equation}
2 m \bar{u}_{s^{\prime}}\left(\mathbf{p}^{\prime}\right) \gamma^{\mu} u_{s}(\mathbf{p})=\bar{u}_{s^{\prime}}\left(\mathbf{p}^{\prime}\right)\left[\left(p^{\prime}+p\right)^{\mu}-2 i S^{\mu \nu}\left(p^{\prime}-p\right)_{\nu}\right] u_{s}(\mathbf{p}).
\end{equation}
Sandwiching instead between $\bar{v}_{s^\prime}(\vb{p}^\prime)$ and $v_s(\vb{p})$, referring again to (\ref{uv}) and (\ref{uv_bar}) gives
\begin{equation}
-2 m \bar{v}_{s^{\prime}}\left(\mathbf{p}^{\prime}\right) \gamma^{\mu} v_{s}(\mathbf{p})=\bar{v}_{s^{\prime}}\left(\mathbf{p}^{\prime}\right)\qty[\left(p^{\prime}+p\right)^{\mu}-2 i S^{\mu \nu}\left(p^{\prime}-p\right)_{\nu}] v_{s}(\mathbf{p}).
\end{equation}
These are known as \textit{Gordon Identities}. A special case is that of $p^\prime=p$:
\begin{equation}
\begin{aligned}
\bar{u}_{s^{\prime}}(\mathbf{p}) \gamma^{\mu} u_{s}(\mathbf{p})&=2 p^{\mu} \delta_{s^{\prime} s} \\
\bar{v}_{s^{\prime}}(\mathbf{p}) \gamma^{\mu} v_{s}(\mathbf{p})&=2 p^{\mu} \delta_{s^{\prime} s}
\end{aligned}
\label{special_gordon_id}
\end{equation}
And we can also show \cite{srednicki2007quantum} that
\begin{equation}
\begin{aligned}
\bar{u}_{s^{\prime}}(\mathbf{p}) \gamma^{0} v_{s}(-\mathbf{p})&=0 \\
\bar{v}_{s^{\prime}}(\mathbf{p}) \gamma^{0} u_{s}(-\mathbf{p})&=0.
\end{aligned}
\label{special_gordon_id_negative}
\end{equation}

As for spin sums, we have
\begin{equation}
\begin{aligned}
\sum_{s=\pm} u_{s}(\mathbf{p}) \bar{u}_{s}(\mathbf{p})&=-\slashed{p}+m \\
\sum_{s=\pm} v_{s}(\mathbf{p}) \bar{v}_{s}(\mathbf{p})&=-\slashed{p}-m
\end{aligned}
\label{spin_sum}
\end{equation}
extrapolated from the rest frame case, where these reduce to  $\sum_{s=\pm} u_{s}(\mathbf{0}) \bar{u}_{s}(\mathbf{0})=m \gamma^{0}+m$ and $\sum_{s=\pm} v_{s}(\mathbf{0}) \bar{v}_{s}(\mathbf{0})=m \gamma^{0}-m$, which we can verify from relations (\ref{uv_columm}) and (\ref{uv_row}).
\section{Creation and annihilation operators}
To express the creation/annihilation operators in terms of the Dirac field we note from (\ref{dirac_mode_expansion}) that   
\begin{equation}
\int \dd^{3} x\, e^{-i p x} \Psi(x)=\sum_{s^{\prime}=\pm}\left[\frac{1}{2 \omega} b_{s^{\prime}}(\mathbf{p}) u_{s^{\prime}}(\mathbf{p})+\frac{1}{2 \omega} e^{2 i \omega t} d_{s^{\prime}}^{\dagger}(-\mathbf{p}) v_{s^{\prime}}(-\mathbf{p})\right].
\end{equation}
We multiply on the left by $\bar{u}_s(\vb{p})\gamma^0$ and use the first relation from  (\ref{special_gordon_id}) and the first relation from (\ref{special_gordon_id_negative}), getting
\begin{equation}
b_{s}(\mathbf{p})=\int \dd^{3} x\, e^{-i p x} \bar{u}_{s}(\mathbf{p}) \gamma^{0} \Psi(x).
\end{equation}
Hermitian conjugation gives $b^\dagger_s(\vb{p})$. Noting that $[\bar{u}_s(\vb{p})\gamma^0\Psi(x)]^\dagger=\Psi^\dagger\gamma^0\bar{u}_s^\dagger(\vb{p})=\bar{\Psi}\beta\gamma^0\beta u_s(\vb{p})=\bar{\Psi}(x)\gamma^0u_s(\vb{p})$ we find
\begin{equation}
b_{s}^\dagger(\mathbf{p})=\int \dd^{3} x\, e^{i p x} \bar{\Psi}(x)\gamma^0u_s(\mathbf{p})
\label{bdagger_psi}
\end{equation}
Similarly, calculating $\int \dd^{3} x\, e^{i p x} \Psi(x)$, multiplying in the left by $\bar{v}_s(\vb{p})\gamma^0$ using the second relation from (\ref{special_gordon_id}) and the second relation from (\ref{special_gordon_id_negative}) gives
\begin{equation}
d^\dagger_{s}(\mathbf{p})=\int \dd^{3} x\, e^{i p x} \bar{v}_{s}(\mathbf{p}) \gamma^{0} \Psi(x),
\end{equation}
and upon hermitian conjugation, we find
\begin{equation}
d_{s}(\mathbf{p})=\int \dd^{3} x\, e^{-i p x} \bar{\Psi}(x)\gamma^0v_s(\mathbf{p}).
\end{equation}

Now we turn to the anti-commutation relations the creation and annihilation operators must satisfy. It is clear from (\ref{anticomm_dirac}) that
\begin{equation}
\begin{array}{l}
\left\{b_{s}(\mathbf{p}), b_{s^{\prime}}\left(\mathbf{p}^{\prime}\right)\right\}=0 \\
\left\{d_{s}(\mathbf{p}), d_{s^{\prime}}\left(\mathbf{p}^{\prime}\right)\right\}=0 \\
\{b_{s}(\mathbf{p}), d_{s^{\prime}}^{\dagger}\left(\mathbf{p}^{\prime}\right)\}=0.
\end{array}
\end{equation}
Hermitian conjugation of these give
\begin{equation}
\begin{aligned}
\{b_{s}^{\dagger}(\mathbf{p}), b_{s^{\prime}}^{\dagger}\left(\mathbf{p}^{\prime}\right)\}&=0 \\
\{d_{s}^{\dagger}(\mathbf{p}), d_{s^{\prime}}^{\dagger}\left(\mathbf{p}^{\prime}\right)\}&=0 \\
\{b_{s}^{\dagger}(\mathbf{p}), d_{s^{\prime}}\left(\mathbf{p}^{\prime}\right)\}&=0.
\end{aligned}
\end{equation}
Now, making use of $(\gamma^0)^2=1$, relations (\ref{special_gordon_id}) and  (\ref{special_gordon_id_negative}) and the canonical anti-commutation relations (\ref{anticomm_dirac}), we find
\begin{equation}
    \begin{aligned}
    \{b_{s}(\mathbf{p}), b^\dagger_{s^{\prime}}\left(\mathbf{p}^{\prime}\right)\}&=(2\pi)^3\delta^3(\vb{p}-\vb{p}^\prime)2\omega\delta_{ss^\prime} \\
    \{d^\dagger_{s}(\mathbf{p}), d_{s^{\prime}}\left(\mathbf{p}^{\prime}\right)\}&=(2\pi)^3\delta^3(\vb{p}-\vb{p}^\prime)2\omega\delta_{ss^\prime}\\
    \{b_{s}(\mathbf{p}), d_{s^{\prime}}\left(\mathbf{p}^{\prime}\right)\}&=0.
    \end{aligned}
    \label{coef_anticomm}
\end{equation}
These are the anti-commutation relations for fermion creation and annihilation operators. We have two types $b$-type fermions and $d$-type fermions and they will be distinguished by their charges and recognized as the electron and the positron, respectively, if $\Psi$ is used as the electron field.
\section{The Dirac field Hamiltonian}
Now we calculate the Hamiltonian density for the Dirac Field
\begin{equation}
    \mathcal{H}=\pdv{\mathcal{L}}{(\partial_0\Psi)}\partial_0\Psi+\pdv{\mathcal{L}}{(\partial_0\bar{\Psi})}\partial_0\bar{\Psi}-\mathcal{L}=-i\bar{\Psi}\gamma^j\partial_j\Psi+m\bar{\Psi}\Psi.
\end{equation}
The total Hamiltonian is
\begin{equation}
    H=\int\dd^3 x\, \Bar{\Psi}(-i\gamma^j\partial_j+m)\Psi
\end{equation}
Noting that
\begin{equation}
    (-i\gamma^j\partial_j+m)\Psi=\sum_{\pm}\int\widetilde{\dd p}\qty[b_s(\vb{p})(\gamma^jp_j+m)u_s(\vb{p})e^{ipx}+d^\dagger_{s}(\vb{p})(-\gamma^jp_j+m)v_s(\vb{p})e^{-ipx}],
\end{equation}
that $(\gamma^jp_j+m)=(\gamma^\mu p_\mu+m-\gamma^0p_0)$, and using (\ref{uv}) gives
\begin{equation}
    (-i\gamma^j\partial_j+m)\Psi=\sum_{\pm}\int\widetilde{\dd p}\qty[b_s(\vb{p})\gamma^0\omega u_s(\vb{p})e^{ipx}-d^\dagger_{s}(\vb{p})\gamma^0\omega v_s(\vb{p})e^{-ipx}].
\end{equation}
Therefore
\begin{equation}
\begin{aligned}
     H&=\sum_{s,s^\prime=\pm}\int\widetilde{\dd p}\,\widetilde{\dd p}^\prime\dd^3 x \qty(b^\dagger_{s^\prime}(\vb{p}^\prime)\bar{u}_{s^\prime}(\vb{p}^\prime)e^{-ip^\prime x}+d_{s^\prime}(\vb{p}^\prime)\bar{v}_{s^\prime}(\vb{p}^\prime)e^{ip^\prime x})\\&\qquad\qquad\times\omega\qty(b_s(\vb{p})\gamma^0 u_s(\vb{p})e^{ipx}-d^\dagger_{s}(\vb{p})\gamma^0 v_s(\vb{p})e^{-ipx})
\end{aligned}
\end{equation}
Working this out gives
\begin{equation}
    \begin{aligned}
        H&=\frac{1}{2}\sum_{s,s^\prime=\pm}\int\widetilde{\dd p}\Big{[}b^\dagger_{s^\prime}(\vb{p})b_s(\vb{p})\overbrace{\bar{u}_{s^\prime}(\vb{p})\gamma^0 u_s(\vb{p})}^{2\omega}
        -b^\dagger_{s^\prime}(-\vb{p})d^\dagger_s(\vb{p})\overbrace{\bar{u}_{s^\prime}(-\vb{p})\gamma^0 v_s(\vb{p})}^{0}e^{2i\omega t}\\
        &\qquad+ d_{s^\prime}(-\vb{p})b_s(\vb{p})\underbrace{\bar{v}_{s^\prime}(-\vb{p})\gamma^0u_s(\vb{p})}_{0}e^{-2i\omega t} -d_{s^\prime}(\vb{p})d^\dagger_s(\vb{p})\underbrace{\bar{v}_{s^\prime}(\vb{p})\gamma^0v_s(\vb{p})}_{2\omega}\Big{]}\\
        &=\sum_s\int\widetilde{\dd p}\,\omega\Big{[}b_s^\dagger(\vb{p})b_s(\vb{p})-d_s(\vb{p})d_s^\dagger(\vb{p})\Big{]}   
    \end{aligned}
\end{equation}
where the over- and under-braces come from (\ref{special_gordon_id}) and (\ref{special_gordon_id_negative}). Using the second anti-commutation relation of (\ref{coef_anticomm}), we find
\begin{equation}
    H=\sum_{s=\pm}\int\widetilde{\dd p}\,\omega\qty[b_s^\dagger(\vb{p})b_s(\vb{p})+d_s^\dagger(\vb{p})d_s(\vb{p})]-4\mathcal{E}_0V
\end{equation}
where $\mathcal{E}_0=\frac{1}{2}\frac{1}{(2\pi)^3}\int\dd^3 k\, \omega$ is the zero point energy and $V=(2\pi)^3\delta^3(\vb{0})$ is the volume of space. To make the ground state energy equal to zero we add $\Omega_0=-4\mathcal{E}_0$ to the Lagrangian $\mathcal{L}$ so that the Hamiltonian is simply\begin{equation}
    H=\sum_{s=\pm}\int\widetilde{\dd p}\,\omega\qty[b_s^\dagger(\vb{p})b_s(\vb{p})+d_s^\dagger(\vb{p})d_s(\vb{p})].
\end{equation}
The Hamiltonian ground state vanishes upon the action of $b$- and $d$-type annihilation operators: $b_s(\vb{p})\ket{0}=d_s(\vb{p})\ket{0}=0$. While  $b^\dagger_s(\vb{p})$ creates an electron with $S_z$ eigenvalue $\frac{1}{2}s$ and $d^\dagger_s(\vb{p})$ creates a positron with $S_z$ eigenvalue $\frac{1}{2}s$. Similarly to the Hamiltonian, we can express the Noether charge in terms of the $b$ and $d$ operators, up to a constant term:
\begin{equation}
    Q=\int\dd^3 x\, \bar{\Psi}\gamma^0\Psi=\sum_{s=\pm}\int\widetilde{\dd p}\qty[b_s^\dagger(\vb{p})b_s(\vb{p})-d_s^\dagger(\vb{p})d_s(\vb{p})]
\end{equation}



