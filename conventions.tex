\chapter*{Notations and Conventions}
\addcontentsline{toc}{chapter}{Notation and Conventions}
We highlight some notation and conventions adopted throughout this text. The set of coordinates of Minkowski's spacetime is denoted by $x=(x^0,x^1,x^2,x^3)$, where $x^0=ct$ and our metric tensor has the signature of mostly positive terms
\begin{equation*}
    g_{\mu\nu}=g^{\mu\nu}=\text{diag}(-1,+1,+1,+1).
\end{equation*}
Repeated indices hide an implicit sum -- Einstein's summation convention. This means, for instance, that
\begin{equation*}
    p_ix^i\equiv\sum_i p_ix^i.
\end{equation*}
Latin indices run from 1 to 3 when denoting space coordinates; or from 1 to 2 when denoting Weyl spinor components; Greek indices run from 0 to 3 when denoting spacetime components;  or from 1 to 4 when denoting Dirac and Majorana bispinor components. With our choice of metric, lowering Lorentz indices of, say, $x^\mu=(x^0,\vb{x})$ flips the sign of the time coordinate, that is
\begin{equation*}
    x_\mu=g_{\mu\nu}x^\nu=(-x^0,\vb{x}),
\end{equation*}
so scalar products reads, for instance,
\begin{equation*}
    px\equiv g_{\mu\nu}p^\mu x^\nu =-p^0x^0+\vb{p}\vdot\vb{x},
\end{equation*}
and the four-momentum is normalized as $p^2=p_\mu p^\mu=-m^2$ in the system of units we adopt throughout the text: $c=\hbar=1$.
The four-gradient reads $\partial_\mu=(\partial_t,\grad)$ and the D'Alambertian operator reads $\partial^2=\partial_\mu\partial^\mu=-\partial_t^2+\laplacian$.\\

Our Fourier transforms will always carry the $2\pi$ factors in the momentum integral. Spacetime transforms and inverse transforms read
\begin{equation*}
    \widetilde f({k})=\int{\dd ^4x} {f}({x})e^{-i{k}{x}},
\end{equation*}
\begin{equation*}
    f({x})=\int\frac{\dd ^4 k}{(2\pi)^4} \widetilde{f}({k})e^{i{k}{x}},
\end{equation*}
while space transforms and inverse transforms read
\begin{equation*}
    \widetilde f({\vb{k}})=\int{\dd ^3x} {f}(\vb{x})e^{-i{\vb{k}}\vdot\vb{x}},
\end{equation*}
\begin{equation*}
    f(\vb{x})=\int\frac{\dd ^3 k}{(2\pi)^3} \widetilde{f}(\vb{k})e^{i\vb{k}\vdot\vb{x}}.
\end{equation*}
With such convention, the Dirac Delta distribution is represented in $d$ dimensions by
\begin{equation*}
    \delta^d(x)=\int\frac{\dd^dk}{(2\pi)^d}e^{ikx}.
\end{equation*}

When dealing with spinors, our Dirac $\gamma$ matrices satisfy
\begin{equation*}
    \pb{\gamma^\mu}{\gamma^\nu}=-2g^{\mu\nu},
\end{equation*}
and we use the chiral -- or Weyl -- representation
\begin{equation*}
\gamma^{0}=\left(\begin{array}{cc}
0 & I \\
I & 0
\end{array}\right), \quad \gamma^{i}=\left(\begin{array}{cc}
0 & \sigma^{i} \\
-\sigma^{i} & 0
\end{array}\right),
\end{equation*}
where $\sigma^i$ are the Pauli Matrices
\begin{equation*}
\sigma^{1}=\left(\begin{array}{ll}
0 & 1 \\
1 & 0
\end{array}\right), \quad \sigma^{2}=\left(\begin{array}{cc}
0 & -i \\
i & 0
\end{array}\right), \quad \sigma^{3}=\left(\begin{array}{cc}
1 & 0 \\
0 & -1
\end{array}\right).
\end{equation*}
For electrodynamics, we adopt Heaviside-Lorentz units. The electron charge $e$ is negative and related to the fine structure constant by $\alpha=e^2/(4\pi)$, since we also take $c=\hbar=1$.\\

Ubiquitously in the text, we use the invariant measure
\begin{equation*}
    \widetilde{\dd p}=\frac{\dd^3 p}{(2\pi)^32\omega}
\end{equation*}
for particles with $p^\mu=(\omega,\vb{p})$.