\chapter{Spinor Fields}
Of the representations of the Lorentz Group, we are already familiarized with the scalars, from our treatment of scalar fields. Now we deal with Weyl spinor fields: fields living in the vector spaces of the $(2,1)$ and $(1,2)$ representations.
\section{Left-handed spinors}
The $({2},1)$ representation of the Lorentz Group consists of $2\times2$ Lorentz transformation matrices acting on a 2-dimensional vector space of fields $\psi_a(x)$, where the spinor index $a$ can take two values.
Lorentz transformations for these spinors read
\begin{equation}
    U^{-1}(\Lambda)\psi_a(x)U(\Lambda)=\tend{L}{a}{b}\psi_b(\Lambda^{-1}x)
\end{equation}
with $\tend{L}{a}{b}(\Lambda)$ being a representation for finite Lorentz transformations, closed under association. Infinitesimal Lorentz transformations are implemented by generators $\tend{(S^{\mu\nu}_L)}{a}{b}$ via
\begin{equation}
    \tend{L}{a}{b}(1+\delta\omega)=\tend{\delta}{a}{b}+\frac{i}{2}\delta\omega_{\mu\nu}\tend{(S^{\mu\nu}_L)}{a}{b}
\end{equation}
where $\tend{(S^{ij}_L)}{a}{b}=\frac{1}{2}\epsilon^{ijk}\sigma_k$ and $\tend{(S^{k0}_L)}{a}{b}=\frac{i}{2}\sigma_k$, $\sigma_k$ being the Pauli Matrices.
\begin{equation}
\sigma_{1}=\left(\begin{array}{ll}
0 & 1 \\
1 & 0
\end{array}\right), \quad \sigma_{2}=\left(\begin{array}{cc}
0 & -i \\
i & 0
\end{array}\right), \quad \sigma_{3}=\left(\begin{array}{cc}
1 & 0 \\
0 & -1
\end{array}\right).
\end{equation}

To justify these claims, we note that (\ref{commuta_generator}) for the spinor generators at the origin (thus neglecting orbital angular momentum) reads
\begin{equation}
    \comm{\psi_A(0)}{M^{\mu\nu}}=\tend{(S^{\mu\nu}_L)}{a}{b}\psi_b(0).
    \label{defining_spinors}
\end{equation}
Recalling that $M^{ij}=\epsilon^{ijk}J_k$, that $(2,1)$ representation has the highest $j$ given by $j=\frac{1}{2}+0=\frac{1}{2}$, and that, in this case, the representation for $J_k$ is $J_k=\frac{1}{2}\sigma_k$, then $\tend{(S^{ij}_L)}{a}{b}=\frac{1}{2}\epsilon^{ijk}\sigma_k$  follows. Noting that in the $(2,1)$ representation $n^\prime=0$, so $N_k^\dagger$ has the trivial scalar representation $N_k^\dagger=0$, then, from (\ref{Ni}) and (\ref{Nidagger}), $K_k=i(N_k-N_k^\dagger)=iN_k=iJ_k$, thus justifying that $\tend{(S^{k0}_L)}{a}{b}=\frac{i}{2}\sigma_k$.
\section{Right-handed spinors}
Hermitian conjugation of a left-handed spinor $(\psi_a)^\dagger$ takes us from the $(2,1)$ representation to the $(1,2)$ representation: that of the right-handed spinors $\psi^\dagger_{\dot{a}}$, signaled by the ``dagger" and the use of dotted indices.
Lorentz transformations for these spinors read
\begin{equation}
    U^{-1}(\Lambda)\psi^\dagger_{\dot{a}}(x)U(\Lambda)=\tend{R}{\dot{a}}{\dot{b}}\psi^\dagger_{\dot{b}}(\Lambda^{-1}x)
\end{equation}
with $\tend{R}{\dot{a}}{\dot{b}}(\Lambda)$ being a representation for finite Lorentz transformations in the $(1,2)$ representation, closed under association.
Infinitesimal Lorentz transformations are implemented by generators $\tend{(S^{\mu\nu}_R)}{\dot{a}}{\dot{b}}$ via
\begin{equation}
    \tend{R}{\dot{a}}{\dot{b}}(1+\delta\omega)=\tend{\delta}{\dot{a}}{\dot{b}}+\frac{i}{2}\delta\omega_{\mu\nu}\tend{(S^{\mu\nu}_R)}{\dot{a}}{\dot{b}}
\end{equation}
where the generator $\tend{(S^{\mu\nu}_R)}{\dot{a}}{\dot{b}}$ is related to $\tend{(S^{\mu\nu}_L)}{a}{b}$ by
\begin{equation}
    \tend{(S^{\mu\nu}_R)}{\dot{a}}{\dot{b}}=-\qty[\tend{(S^{\mu\nu}_L)}{{a}}{{b}}]^*
\end{equation}
as can be seen from hermitian conjugation of (\ref{defining_spinors}).
\section{Two-spinor indices fields}
We now consider fields living in the vector space of the $(2,1)\otimes(2,1)$ representation. From our knowledge of angular momentum addition, we know that $(2,1)\otimes(2,1)=(1,1)\oplus(3,1)$ where the $(1,1)$ is the antisymmetric scalar (singlet) representation and $(3,1)$ is the symmetric triplet representation. This allows the decomposition
\begin{equation}
    C_{ab}(x)=\epsilon_{ab}D(x)+G_{ab}(x)
    \label{decompo}
\end{equation}
where $D(x)$ is a scalar, $\epsilon_{ab}=-\epsilon_{ba}$ is the two-dimensional antisymmetric Levi-Civita symbol, and $G_{ab}$ is a symmetric field.
The fact that $D(x)$ is a scalar implies that $\epsilon_{ab}$ is a Lorentz invariant symbol, since transformation of (\ref{decompo}) will lead to $\tend{L}{a}{c}(\Lambda)\tend{L}{b}{d}(\Lambda)\epsilon_{cd}=\epsilon_{ab}$. \\
%Expansion of the latter expression in infinitesimal transformations reveal the antisymmetry of the generators ${(S_L^{\mu\nu})}_{ab}$ and ${(S_R^{\mu\nu})}_{\dot{a}\dot{b}}$ \cite{srednicki2007quantum}.\\

Similarly, if we consider the representation $(1,2)\otimes(1,2)=(1,1)\oplus(1,3)$, we can have the decomposition of fields
\begin{equation}
    C_{\dot{a}\dot{b}}(x)=\epsilon_{\dot{a}\dot{b}}D(x)+G_{\dot{a}\dot{b}}(x)
\end{equation}
where, again $G_{\dot{a}\dot{b}}(x)$ is symmetric, $D(x)$ is a scalar and $\epsilon_{\dot{a}\dot{b}}=-\epsilon_{\dot{b}\dot{a}}$ is the invariant antisymmetric Levi-Civita symbol, with dotted indices this time. The invariant Levi-Civita symbol is analogous to the metric tensor. The metric allows the connection between vectors and their duals, that is, allows the lowering or raising of indices. Similarly, the Levi-Civita symbol allows to write
\begin{equation}
    \begin{aligned}
    \psi_a&=\epsilon_{ab}\psi^b\\
    \psi^a&=\epsilon^{ab}\psi_b
    \end{aligned}
\end{equation}
and
\begin{equation}
    \begin{aligned}
    \psi_{\dot{a}}&=\epsilon_{\dot{a}\dot{b}}\psi^{\dot{b}}\\
    \psi^{\dot{a}}&=\epsilon^{\dot{a}\dot{b}}\psi_{\dot{b}}.
    \end{aligned}
\end{equation}
The difference from the usual contraction, though, is that with spinors we can contract on the second index only. For consistency, we must have
\begin{equation}
    \begin{aligned}
    \epsilon^{{1}{2}}&=\epsilon_{{2}{1}}=1\\
    \epsilon^{{2}{1}}&=\epsilon_{{1}{2}}=-1
    \end{aligned}
\end{equation}
\begin{equation}
    \begin{aligned}
    \epsilon^{\dot{1}\dot{2}}&=\epsilon_{\dot{2}\dot{1}}=1\\
    \epsilon^{\dot{2}\dot{1}}&=\epsilon_{\dot{1}\dot{2}}=-1
    \end{aligned}
\end{equation}
and also
\begin{equation}
\begin{aligned}
\epsilon_{ab}\epsilon^{bc}&=\delta_a^c\\
\epsilon^{ab}\epsilon_{bc}&=\delta^a_c
\end{aligned}
\end{equation}
valid also for dotted indices.\\

We now turn to fields with two spinor indices, one of them undotted and the other dotted. These are the fields living in the vector space of the $(2,1)\otimes(1,2)\otimes(2,2)$ representation. This is the vector representation and the decomposition of vectors into irreducible spinor representations induces the existence of an invariant symbol connecting two-spinor indices  fields and the four-vectors 
\begin{equation}
    A_{a\dot{a}}=\sigma^\mu_{a\dot{a}}A_\mu.
\end{equation}
This invariant symbol, with our conventions, is
\begin{equation}
    \sigma^\mu_{a\dot{a}}=(I,\va{\sigma})
    \label{sigma1}
\end{equation}
being $I$ the $2\times2$ identity matrix and $\va{\sigma}=(\sigma_1,\sigma_2,\sigma_3)$, where $\sigma_k$ are the Pauli matrices \cite{srednicki2007quantum}.
Important relations satisfied by this invariant symbol are
\begin{equation}
    \sigma^\mu_{a\dot{a}}\sigma_{\mu b\dot{b}}=-2\epsilon_{ab}\epsilon_{\dot{a}\dot{b}}
\end{equation}
\begin{equation}
    \epsilon^{ab}\epsilon^{\dot{a}\dot{b}}\sigma^\mu_{a\dot{a}}\sigma^\nu_{b\dot{b}}=-2g^{\mu\nu}
\end{equation}
which can be verified by inspection.
Raising spinor indices of the invariant symbol gives
\begin{equation}
    \bar{\sigma}^{\mu\dot{a}a}\equiv\epsilon^{ab}\epsilon^{\dot{a}\dot{b}}\sigma^\mu_{b\dot{b}}=(I,-\va{\sigma}).
    \label{sigma2}
\end{equation}
Lorentz invariance of $\sigma^\mu_{a\dot{a}}$ means that
\begin{equation}
    \sigma^\mu_{a\dot{a}}=\ten{\Lambda}{\mu}{\nu}\tend{L}{a}{b}(\Lambda)\tend{R}{\dot{a}}{\dot{b}}(\Lambda)\sigma^\nu_{b\dot{b}}
\end{equation}
with each index undergoing a transformation matrix according to its representation. Expansion of this expression in terms of infinitesimal transformations leads to \cite{srednicki2007quantum}
\begin{equation}
    \tend{(S^{\mu\nu}_L)}{a}{b}=\frac{i}{4}\tend{(\sigma^\mu\bar{\sigma}^\nu-\sigma^\nu\bar{\sigma}^\mu)}{a}{b}.
    \label{SL}
\end{equation}
and
\begin{equation}
    \ten{(S^{\mu\nu}_R)}{\dot{a}}{\dot{b}}=-\frac{i}{4}\ten{(\bar{\sigma}^\mu{\sigma}^\nu-\bar{\sigma}^\nu{\sigma}^\mu)}{\dot{a}}{\dot{b}}.
    \label{SR}
\end{equation}
where pairs of contracted spinor indices are omitted, a convention we adopt from now on. The missing pairs of contracted undotted indices must be interpreted as $\ten{\phantom{A}}{c}{c}$, while as $\tend{\phantom{A}}{\dot{c}}{\dot{c}}$  for the dotted indices, that is: $\xi\psi=\xi^a\psi_a$ and $\xi^\dagger\psi^\dagger=\xi^\dagger_{\dot{a}}\psi^{\dagger\dot{a}}$

\section{Lagrangians and equations of motion for spinor fields}
When looking for Lagrangians for spinor fields, we would like them to be Hermitian, Lorentz invariant and quadratic so their equations of motion are linear, thus accommodating plane-wave solutions needed to describe free particles. A candidate is
\begin{equation}
    \mathcal{L}=i\psi^\dagger\bar{\sigma}^\mu\partial_\mu\psi-\frac{1}{2}m\psi\psi-\frac{1}{2}m\psi^\dagger\psi^\dagger.
\end{equation}
where the first term is not strictly hermitian, but its hermitian conjugate differs from the term itself only by a 4-divergence, which can be made to vanish with appropriate boundary conditions. With an action $S=\int\dd^4x\,\mathcal{L}$, the equations of motion follow from the variational principle
\begin{equation}
    -\fdv{S}{\psi^\dagger}=-i\bar{\sigma}^\mu\partial_\mu\psi+m\psi^\dagger=0
\end{equation}
\begin{equation}
    -\fdv{S}{\psi}=-i{\sigma}^\mu\partial_\mu\psi^\dagger+m\psi=0
\end{equation}
which can be condensed by
\begin{equation}
\left(\begin{array}{cc}
m \delta_{a}^{\phantom{a}c} & -i \sigma_{a \dot{c}}^{\mu} \partial_{\mu} \\
-i \bar{\sigma}^{\mu \dot{a} c} \partial_{\mu} & m \delta^{\dot{a}}_{\phantom{a}\dot{c}}
\end{array}\right)\left(\begin{array}{c}
\psi_{c} \\
\psi^{\dagger \dot{c}}
\end{array}\right)=0.
\label{dirac_equation_matrix}
\end{equation}
Introducing the \textit{gamma matrices}
\begin{equation}
\gamma^{\mu} \equiv\left(\begin{array}{cc}
0 & \sigma_{a \dot{c}}^{\mu} \\
\bar{\sigma}^{\mu \dot{a} c} & 0
\end{array}\right)
\end{equation}
and the \textit{Majorana Field}
\begin{equation}
\Psi \equiv\left(\begin{array}{c}
\psi_{c} \\
\psi^{\dagger \dot{c}}
\end{array}\right)
\end{equation}
equation (\ref{dirac_equation_matrix}) is recognized as the \textit{Dirac Equation}
\begin{equation}
    (-i\gamma^\mu\partial_\mu+m)\Psi=0.
\end{equation}
From the relations among the $\sigma$ matrices
\begin{equation}
\begin{array}{l}
\left(\sigma^{\mu} \bar{\sigma}^{\nu}+\sigma^{\nu} \bar{\sigma}^{\mu}\right)_{a}^{\phantom{a}c}=-2 g^{\mu \nu} \delta_{a}^{\phantom{a}c} \\
\left(\bar{\sigma}^{\mu} \sigma^{\nu}+\bar{\sigma}^{\nu} \sigma^{\mu}\right)^{\dot{a}}_{\phantom{a}\dot{c}}=-2 g^{\mu \nu} \delta^{\dot{a}}_{\phantom{a}\dot{c}}
\end{array}
\end{equation}
which can be verified using (\ref{sigma1}) and (\ref{sigma2}), follows that the $\gamma$ matrices satisfy
\begin{equation}
    \pb{\gamma^\mu}{\gamma^\nu}=-2g^{\mu\nu}I
\end{equation}
where $\pb{\gamma^\mu}{\gamma^\nu}=\gamma^\mu\gamma^\nu+\gamma^\nu\gamma^\mu$ is the anti-commutator and $I$ is the $4\times4$ identity matrix.\\

If we now consider the Lagrangian for two spinor fields $\psi_1$ and $\psi_2$
\begin{equation}
    \mathcal{L}=i\psi^\dagger_i\bar{\sigma}^\mu\partial_\mu\psi_i-\frac{1}{2}m\psi_i\psi_i-\frac{1}{2}m\psi_i^\dagger\psi_i^\dagger
\end{equation}
we can see that this Lagrangian is SO(2) invariant, that is, upon the transformation
\begin{equation}
\left(\begin{array}{l}
\psi_{1} \\
\psi_{2}
\end{array}\right) \rightarrow\left(\begin{array}{rr}
\cos \alpha & \sin \alpha \\
-\sin \alpha & \cos \alpha
\end{array}\right)\left(\begin{array}{l}
\psi_{1} \\
\psi_{2}
\end{array}\right) .
\end{equation}
we see that $\mathcal{L}\to\mathcal{L}$. This rotation invariance can be also highlighted by defining 
\begin{equation}
    \begin{aligned}
    \chi&=\frac{1}{\sqrt{2}}(\psi_1+i\psi_2)\\
    \xi&=\frac{1}{\sqrt{2}}(\psi_1-i\psi_2)
    \end{aligned}
\end{equation}
and expressing the Lagrangian as
\begin{equation}
    \mathcal{L}=i\chi^\dagger\bar{\sigma}^\mu\partial_\mu\chi+i\xi^\dagger\bar{\sigma}^\mu\partial_\mu\xi-m\chi\xi-m\chi^\dagger\xi^\dagger
    \label{dirac_lagrangian_weyl}
\end{equation}
with a manifestly U(1) invariance
\begin{equation}
    \begin{aligned}
    \chi&\to e^{-i\alpha}\chi\\
    \xi&\to e^{i\alpha}\xi\\
    \mathcal{L}&\to\mathcal{L}.
    \end{aligned}
\end{equation}
The equations of motion for fields $\chi$ and $\xi$ are
\begin{equation}
\left(\begin{array}{cc}
m \delta_{a}^{\phantom{a}c} & -i \sigma_{a \dot{c}}^{\mu} \partial_{\mu} \\
-i \bar{\sigma}^{\mu \dot{a} c} \partial_{\mu} & m \delta^{\dot{a}}_{\phantom{a}\dot{c}}
\end{array}\right)\left(\begin{array}{c}
\chi_{c} \\
\xi^{\dagger \dot{c}}
\end{array}\right)=0.
\end{equation}
Defining the \textit{Dirac Field}
\begin{equation}
\Psi \equiv\left(\begin{array}{c}
\chi_{c} \\
\xi^{\dagger \dot{c}}
\end{array}\right)
\end{equation}
again we have the Dirac equation as the equation of motion
\begin{equation}
    (-i\gamma^\mu\partial_\mu+m)\Psi(x)=0.
\end{equation}

The Lagrangian for the Dirac Field can be cast in terms of the Dirac Field itself, rather than in terms of its Weyl right-handed and left-handed spinor components. It reads
\begin{equation}
    \mathcal{L}=i\bar{\Psi}\gamma^\mu\partial_\mu\Psi-m\bar{\Psi}\Psi
    \label{dirac_lagrangian}
\end{equation}
where we introduced
\begin{equation}
    \bar{\Psi}=\Psi^\dagger\beta
\end{equation}
being $\Psi^\dagger=(\chi^\dagger_{\dot{a}}\,,\,\xi^a)$ and $\beta$ the matrix
\begin{equation}
\beta \equiv\left(\begin{array}{cc}
0 & \delta^{\dot{a}}{ }_{\dot{c}} \\
\delta_{a}^{\phantom{a}c} & 0
\end{array}\right)
\end{equation}
which is numerically equal to $\gamma^0$, but different in spinor indices placement. This means
\begin{equation}
\bar{\Psi}=(\xi^a,\chi^\dagger_{\dot{a}}).    
\end{equation}
Strictly speaking, (\ref{dirac_lagrangian}) differs from (\ref{dirac_lagrangian_weyl}) by a 4-divergence, which can be made to vanish for appropriate boundary conditions. Lagrangian (\ref{dirac_lagrangian}) also manifests U(1) invariance, that is
\begin{equation}
    \begin{aligned}
    \Psi&\to e^{-i\alpha}\Psi\\
    \bar{\Psi}&\to e^{i\alpha}\bar{\Psi}
    \end{aligned}
\end{equation}
leaves the Lagrangian invariant. This means there is a conserved Noether current
\begin{equation}
    j^\mu=\bar{\Psi}\gamma^\mu\Psi=\chi^\dagger\bar{\sigma}^\mu\chi-\xi^\dagger\bar{\sigma}^\mu\xi.
\end{equation}

%We can recover the Weyl spinor components from the Dirac or %majorana fields by defining the matrix
%\begin{equation}
%\gamma_5\equiv\left(\begin{array}{cc}
% -\delta_{a}^{\phantom{a}c} & 0\\
%0 &  \delta^{\dot{a}}_{\phantom{a}\dot{c}}
%\end{array}\right)
%\end{equation}
%and the projection operators
%\begin{equation}
%P_L=\frac{1}{2}(I-\gamma_5)=\left(\begin{array}{cc}
% \delta_{a}^{\phantom{a}c} & 0\\
%0 &  0
%\end{array}\right)
%\end{equation}
%\begin{equation}
%P_R=\frac{1}{2}(I+\gamma_5)=\left(\begin{array}{cc}
% 0 & 0\\
%0 &  \delta^{\dot{a}}_{\phantom{a}\dot{c}}
%\end{array}\right)
%\end{equation}

Since the Dirac and Majorana fields are made of Weyl Fields, according to (\ref{SL}) and (\ref{SR}) we can write the Lorentz transformation generator for $\Psi$ as
\begin{equation}
 S^{\mu \nu}\equiv\frac{i}{4}\left[\gamma^{\mu}, \gamma^{\nu}\right]=\left(\begin{array}{cc}
+\left(S_{{L}}^{\mu \nu}\right)_{a}^{\phantom{a}c}& 0 \\
0 & -\left(S_{{R}}^{\mu \nu}\right)^{\dot{a}}_{\phantom{a}\dot{c}}
\end{array}\right) 
\end{equation}
then infinitesimal transformations read
\begin{equation}
D(1+\delta \omega)=1+\frac{i}{2} \delta \omega_{\mu \nu} S^{\mu \nu}
\end{equation}
and finite ones given by
\begin{equation}
U^{-1}(\Lambda) \Psi(x) U(\Lambda)=D(\Lambda) \Psi\left(\Lambda^{-1} x\right).
\end{equation}
