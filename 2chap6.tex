\chapter{The Lorentz Group and its Representations}
\section{Lorentz transformations}
Special relativity tells us  that the fundamental symmetry of spacetime is the \textit{spacetime interval}. For events separated by $\dd t$, $\dd x$ $\dd y$, $\dd z$, according to an observer, and by $\dd t^\prime$, $\dd x^\prime$ $\dd y^\prime$, $\dd z^\prime$, according to a different observer, we will always check
\begin{equation}
    \dd s^2 =  -\dd t^{2}+ \dd x^{2}+ \dd y^{ 2}+\dd z^{2} =-\dd t^{\prime 2}+ \dd x^{\prime2}+ \dd y^{\prime 2}+\dd z^{\prime2}=\dd s^{\prime2}.
\end{equation}
Introducing the metric tensor $g_{\mu\nu}$ we write the interval as
\begin{equation}
    \dd s^2=g_{\mu\nu}\dd x^\mu\dd x^\nu
\end{equation}
with an implicit sum and using $g_{\mu\nu}=\text{diag}(-1,+1,+1,+1)$. Transformations leaving the interval invariant are \textit{Lorentz Transformations}, with matrix representation $\Lambda^\mu_{\phantom{a}\nu}$. Under these, coordinates transform as 
\begin{equation}
    x^\mu\to x^{\prime\mu}=\Lambda^\mu_{\phantom{a}\nu} x^\nu
\end{equation}
and so does any other 4-vector, such as $\dd x^\mu$. Thus, the invariance of the interval is stated by
\begin{equation}
    \dd s^{\prime2}=g_{\mu\nu}\dd x^{\prime\mu}\dd x^{\prime\nu}=g_{\mu\nu}\Lambda^\mu_{\phantom{a}\rho} \Lambda^\nu_{\phantom{a}\sigma}\dd x^\rho x^\sigma= 
    g_{\rho\sigma}\dd x^\rho\dd x^\sigma=\dd s^2
\end{equation}
revealing that the transformations for which the interval is preserved are those satisfying
\begin{equation}
    g_{\mu\nu}\Lambda^\mu_{\phantom{a}\rho} \Lambda^\nu_{\phantom{a}\sigma}=
    g_{\rho\sigma}
    \label{condicao}
\end{equation}
or, in matrix notation,
\begin{equation}
    \Lambda^{\text{T}}g\Lambda=g.
    \label{condicao_matriz}
\end{equation}
Inverse transformations can be readily obtained from (\ref{condicao}). Allowing the metric to lower one of the indices, say, $\mu$ and then raising the other gives $g^{\rho\lambda}\Lambda_{\nu\rho}\Lambda^\nu_{\phantom{a}\sigma}=\Lambda_\nu^{\phantom{a}\lambda}\Lambda^\nu_{\phantom{a}\sigma}=g^{\rho\lambda}g_{\rho\sigma}=\delta^\lambda_\sigma$ telling that the inverse has components $(\Lambda^{-1})^\lambda_{\phantom{a}\nu}=\Lambda_\nu^{\phantom{a}\lambda}$, since multiplication of the latter with $\ten{\Lambda}{\nu}{\sigma}$ results in the identity matrix $\delta^\lambda_\sigma$.\\

Matrix condition (\ref{condicao_matriz}) indicates that $\det \Lambda=\pm1$, since $\det(\Lambda^{\text{T}}g\Lambda)=\det\Lambda^{\text{T}}\det g\det\Lambda=(\det \Lambda)^2\det g=\det g$. Transformations with $\det\Lambda=-1$ correspond to reflections and can be attained by defining a discrete parity transformation. We restrict ourselves to those with $\det\Lambda=+1$, which define the \textit{proper sector} of Lorentz Group. If we further restrict ourselves to transformations with $\ten{\Lambda}{0}{0}\geq +1$, we will be dealing with the \textit{proper and orthochronous} sector of the Lorentz Group. The general group satisfying (\ref{condicao_matriz}) is called O(1,3) and can be recast from the proper and orthochronous Lorentz Group SO(1,3) with the aid of time and parity inversion transformations. Proper and orthochronous transformations are the ones we are interested, since they can be obtained by successive compounding of infinitesimal transformations: those arbitrarily close to the identity transformation. When referring to the Lorentz Group, from now on, we are referring to proper and orthochronous sector.\\

An infinitesimal Lorentz transformation reads
\begin{equation}
    \ten{\Lambda}{\mu}{\nu}=\ten{\delta}{\mu}{\nu}+\ten{\delta\omega}{\mu}{\nu}
\end{equation}
where $\ten{\delta\omega}{\mu}{\nu}$ is must be antisymmetric in order to preserve the interval. Transformations affecting only space components are rotations. A rotation by an angle $\delta\theta$ about a unit vector $\vu{n}$ is generated by
\begin{equation}
    \delta\omega_{ij}=-\epsilon_{ijk}n_k\delta\theta
\end{equation}
where $\epsilon_{ijk}$ is the Levi-Civita antisymmetric symbol, and $n_k$ are the components of $\vu{n}$. On the other hand, proper orthochronous transformations mixing space and time components are caleed \textit{boosts}. A boost along $\vu{n}$ by an amount $\delta\eta$ is generated by
\begin{equation}
    \delta\omega_{i0}=n_i\delta\eta
\end{equation}
where, again, $n_i$ are the components of $\vu{n}$.
\section{Lorentz group representations}
In quantum theory, transformations in coordinates induce unitary and linear transformations to states and fields \cite{weinberg1995quantum}. These are labeled $U(\Lambda)$ and satisfy $U^\dagger=U^{-1}$, i.e, are unitary. They must also be closed under association: $U(\Lambda^\prime\Lambda)=U(\Lambda^\prime)U(\Lambda)$. Infinitesimal transformations are implemented by hermitian generators $M^{\mu\nu}$ via
\begin{equation}
   U(1+\delta\omega)=I+\frac{i}{2}\delta\omega_{\mu\nu}M^{\mu\nu}.
   \label{infinitesimalU}
\end{equation}
The antisymmetry of $\delta\omega_{\mu\nu}$ leads to the antisymmetry of $M^{\mu\nu}$ too, ie $M^{\mu\nu}=-M^{\nu\mu}$. The associative property tells us that
\begin{equation}
    U^{-1}(\Lambda)U(\Lambda^\prime)U(\Lambda)=U(\Lambda^{-1}\Lambda^\prime\Lambda).
    \label{associative}
\end{equation}
Expanding $\Lambda^\prime$ as an infinitesimal transformation and keeping terms linear in $\delta\omega_{\mu\nu}$  gives
\begin{equation}
    U^{-1}(\Lambda)M^{\mu\nu}U(\Lambda)=\ten{\Lambda}{\mu}{\rho}\ten{\Lambda}{\nu}{\sigma}M^{\rho\sigma}
    \label{transform_tensor}
\end{equation}
indicating that the generators transform as tensors. For a vector, such as $P^\mu$, we expect
\begin{equation}
    U^{-1}(\Lambda)P^{\mu}U(\Lambda)=\ten{\Lambda}{\mu}{\nu}P^\nu.
    \label{transform_vector}
\end{equation}
Further expanding $\Lambda=1+\delta\omega$ in (\ref{transform_tensor}), we find
\begin{equation}
    \comm{M^{\mu\nu}}{M^{\rho\sigma}}=i[(g^{\mu\rho}M^{\nu\sigma}-g^{\nu\rho}M^{\mu\sigma})-(g^{\mu\sigma}M^{\nu\rho}-g^{\nu\sigma}M^{\mu\rho})]
    \label{Lorentz_algebra_M}
\end{equation}
defining the algebra of the Lorentz Group. Introducing 
\begin{equation}
    \begin{aligned}
    J_i&=\frac{1}{2}\epsilon_{ijk}M^{jk}\\
    K_i&=M^{i0}
    \end{aligned}
\end{equation}
equation (\ref{Lorentz_algebra_M}) is satisfied if so these are
\begin{equation}
    \begin{aligned}
    \comm{J_i}{J_j}&=i\epsilon_{ijk}J_k\\
    \comm{J_i}{K_j}&=i\epsilon_{ijk}K_k\\
    \comm{K_i}{K_j}&=-i\epsilon_{ijk}J_k.
    \end{aligned}
    \label{lorentz_algebra_JK}
\end{equation}
For (\ref{transform_vector}) expansion $\Lambda=1+\delta\omega$ leads to
\begin{equation}
    \comm{P^\mu}{M^{\rho\sigma}}=i(g^{\mu\sigma}P^\rho-g^{\mu\rho}P^\sigma).
\end{equation}
Since $P^\mu=(H,\vb{P})$ this algebra can be broken down to
\begin{equation}
\begin{aligned}
\left[J_{i}, H\right] &=0 \\
\left[J_{i}, P_{j}\right] &=i \epsilon_{i j k} P_{k} \\
\left[K_{i}, H\right] &=i  P_{i} \\
\left[K_{i}, P_{j}\right] &=i \delta_{i j} H .
\end{aligned}
\end{equation}
These plus the commutation relation of the components of $P^\mu$ with itself
\begin{equation}
\begin{array}{l}
{\left[P_{i}, P_{j}\right]=0,} \\
{\left[P_{i}, H\right]=0 .}
\end{array}
\end{equation}
give the algebra of \textit{Poincaré Group}, for which the generators are the generators of Lorentz transformations and the generator of translations $P^\mu$. The Poincaré group, therefore, encodes the symmetries of spacetime.\\

For a scalar field $\phi$ with a vacuum state $\ket{0}$, Lorentz invariance gives
\begin{equation}
    \vev{\phi(x)}=\ev{\phi(\Lambda x)}{\Lambda 0}=\vev{U^{-1}\phi(\Lambda x)U}
\end{equation}
Meaning $\phi(x)=U^{-1}\phi(\Lambda x)U$, or, equivalently, changing who we call $x$
\begin{equation}
    U^{-1}(\Lambda)\phi(x)U(\Lambda)=\phi(\Lambda^{-1}x).
    \label{transfor_scalar}
\end{equation}
Equations (\ref{transform_tensor}) and (\ref{transform_vector}) show how tensors and vectors transforms, and (\ref{transfor_scalar}) tell us how scalars do.\\

We now consider a general group representation matrix $\tend{L}{A}{B}$  to act on a field $\phi_B$ with arbitrary number of components. Infinitesimal transformations are implemented by generators $\tend{(S^{\mu\nu})}{A}{B}$ via
\begin{equation}
    \tend{L}{A}{B}(I+\delta\omega)=\tend{\delta}{A}{B}+\frac{i}{2}\delta_{\mu\nu}\omega\tend{(S^{\mu\nu})}{A}{B}.
\end{equation}
Then, inspired by (\ref{transform_tensor}), (\ref{transform_vector}) and (\ref{transfor_scalar}), the transformation of a field reads
\begin{equation}
    U^{-1}(\Lambda)\phi_A U(\Lambda)=\tend{L}{A}{B}\phi_B(x)(\Lambda^{-1}x).
    \label{infinitesimalL}
\end{equation}
Expanding both sides of this relation as infinitesimal transformations, using (\ref{infinitesimalU}) for $U(\Lambda)$ and  (\ref{infinitesimalL}) for $\tend{L}{A}{B}$, collecting terms up to first order in $\delta\omega_{\mu\nu}$ gives
\begin{equation}
    \comm{\phi_A(x)}{M^{\mu\nu}}=\mathcal{L}^{\mu\nu}\phi_A(x)+\tend{(S^{\mu\nu})}{A}{B}\phi_B(x)
    \label{commuta_generator}
\end{equation}
where $\mathcal{L}^{\mu\nu}=\frac{1}{i}(x^\mu\partial^\nu-x^\nu\partial^\mu)$ is the generator of orbital angular momentum. Both $\mathcal{L}^{\mu\nu}$ and $\tend{(S^{\mu\nu})}{A}{B}$ have the same commutation relation as the generator $M^{\mu\nu}$ in (\ref{Lorentz_algebra_M}) \cite{srednicki2007quantum, maggiore2005modern}.

The task of identifying the matrices $\tend{(S^{\mu\nu})}{A}{B}$ is the task of determining matrices 
\begin{equation}
    \begin{aligned}
        \tend{(J_i)}{A}{B}&=\frac{1}{2}\epsilon_{ijk}\tend{(S^{jk})}{A}{B}\\
        \tend{(K_i)}{A}{B}&=\tend{(S^{i0})}{A}{B}
    \end{aligned}
\end{equation}
with commutation relations (\ref{lorentz_algebra_JK}). Knowing the generators of a representation we can reconstruct the transformation by compounding (exponentiation). We know from the theory of angular momentum that we can find matrices $(2j+1)\times(2j+1)$  $\mathcal{T}_1$, $\mathcal{T}_2$, and $\mathcal{T}_3$, with eigenvalues of $\mathcal{T}_3$ being $-j,-j+1,\dots,j-1,j$, where $j=0,\frac{1}{2},1,\dots$ satisfying the first relation in (\ref{lorentz_algebra_JK}). Each representation of these matrices is labeled by $j$, and act over $(2j+1)$ dimensional vector spaces. These are the \textit{irreducible} (cannot be made block diagonal) and \textit{inequivalent} (not related by unitary transformations) representations of the Lie Algebra of SO(3), the group of rotations in three dimensions. SO(3) has the same Lie Algebra as SU(2), the group of the unit quaternions or $2\times2$ unitary matrices with unit determinant. For half integer $j$, though, the representations of the Lie algebra are not representations of SO(3) group, since states pick-up a minus sign from a $2\pi$ rotation. On the other hand, half integer representations of algebra SU(2) are actually representations of the group SU(2) and we say that SU(2) is the double cover of SO(3) \cite{schwichtenberg2017physics}.
What we see next is that by considering the complexification of the algebra of SO(1,3), we arrive at two irreducible representations of SU(2) and these will be the representations of the covering group of the Lorentz group \cite{schwichtenberg2017physics}. We define
\begin{equation}
    N_i=\frac{1}{2}(J_i-iK_i),
    \label{Ni}
\end{equation}
\begin{equation}
    N_i^\dagger=\frac{1}{2}(J_i+iK_i),
    \label{Nidagger}
\end{equation}
then algebra (\ref{lorentz_algebra_JK})  becomes
\begin{equation}
    \comm*{N_i}{N_j}=i\epsilon_{ijk}N_k
\end{equation}
\begin{equation}
    \comm*{N_i^\dagger}{N_j^\dagger}=i\epsilon_{ijk}N_k^\dagger
\end{equation}
\begin{equation}
    \comm*{N_i}{N_j^\dagger}=0
\end{equation}
These are two SU(2) algebras, differing by hermitian conjugation. So the representations of SL(2,$\mathbb{C}$), the covering group of SO(1,3) in four spacetime dimensions, are specified by two integers or half-integers $n$ and $n^\prime$ specifying the two SU(2) representations. These are labeled by $(2n+1,2n^\prime+1)$. Alternatively, since $J_i=N_i+N_i^\dagger$, our knowledge of angular momentum addition tell us that $j$ must be constrained to assume $j=\abs{n-n^\prime},\dots,n+n^\prime$, so we can specify the representations also by the highest $j$ they accommodate. We have the following representations for the covering group of the Lorentz Group
\begin{itemize}
    \item Scalar or singlet representation $(2n+1,2n^\prime+1)=(1,1)$: Trivial generators acting on a one-dimensional vector space of single-component scalars
    \item Left-handed spinor representation $(2,1)$: Generators acting on two-dimensional vector space of two-components spinors
    \item Right-handed spinor representation $(1,2)$ :
    Generators acting on two-dimensional vector space of two-components spinors
    \item Vector representation $(2,2)$:
    Generators acting on four-dimensional vector space of 4-component vectors.
\end{itemize}
